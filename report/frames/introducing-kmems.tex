\begin{frame}
	\frametitle{k-MEMs}
	\framesubtitle{k-MEMs: LEFTMAX, RIGHTMAX}
	Let \(Q \in \Sigma^+\) be a query string, \(\kappa\) be a threshold, \(lext(i, P, j)\) the left extension
	of the string \(P[i..j]\) and \(rext(i, P, j)\) the right extension.

	\onslide<2-3>\begin{definition}[LEFTMAX]
		A match \(([x..y], (i, P, j))\) of \(Q[x..y]\) in \(G\) satisfies the \(LEFTMAX\) property
		if and only if
		\[
			x = 1 \ \vee \ lext(i, P, j) = \emptyset \ \vee \ Q[x-1] \notin lext(i, P, j)
		\]
	\end{definition}

	\onslide<3>We can analogously define the RIGHTMAX property.
\end{frame}

\begin{frame}
	\frametitle{k-MEMs}
	\framesubtitle{k-MEMs: Definition}
	\begin{definition}[\(\kappa\)-MEM]
		A match \(([x..y], (i, P, j))\) of \(Q[x..y]\) in \(G\) is called a \(\kappa\)-MEM if it satisfies
		all the following conditions:
		\begin{enumerate}
			\item \(LEFTMAX \vee |lext(i, P, j)| \geq 2 \)
			\item \(RIGHTMAX \vee |rext(i, P, j)| \geq 2 \)
			\item \(y - x + 1 \geq \kappa\)
		\end{enumerate}
	\end{definition}
\end{frame}

